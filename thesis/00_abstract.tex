\selectlanguage{english}
\begin{abstract}
The aim of this thesis is to improve and benchmark \lare{} - A new technology for stateful \singlePageApplication{}s.
\lare{} is a front- and backend technology, developed to easily improve \webSite{}s with the use of \ajax{} but without the disadvantages of it regarding browser functionality, SEO and user experience.
\\
The thesis first presents \lare{} and describes how a realisation of it ideally should be structured. 
In a second stage a PHP backend, called \phpLare{} and the matching Twig extension \twigLare{} are introduced using those concepts. 
Additionally the existing JavaScript frontend \lareJS{} and the django backend \djangoLare{} are refactored considering these guidelines.
Finally \lare{} gets evaluated in form of benchmarks using \curl{} for plain \httpRequest{}s and \selenium{} for retrieving \webPage{}s including all resources. 
For this purpose a sample web application is implemented using PHP, Twig and the matching \lare{} plugins.
\\
The results of this evaluation show that \lare{} satisfies it's expectations regarding browser functionality and improves the load times of \singlePageApplication{}s.
\end{abstract}

\selectlanguage{ngerman}
\begin{abstract}
Ziel dieser Thesis ist es \lare{} - eine neue Technologie f\"ur stateful \singlePageApplication{}s - zu verbessern und auf Geschwindigkeit zu testen.
\lare{} ist eine auf PJAX basierende Front- und Backend Technologie, die es erm\"oglicht ohne gro\ss{}en Aufwand Webseiten mit \ajax{} Unterst\"utzung ohne dessen Nachteile im Bereich Browser Funktionalit\"at, SEO und User Experience zu entwickeln.
\\
Die Thesis pr\"asentiert anfangs \lare{} und beschreibt wie eine Realisierung idealerweise strukturiert sein sollte.
In einem zweiten Schritt werden ein PHP Backend, genannt \phpLare{} und eine passende Twig Extension \twigLare{} eingef\"uhrt, die diese Konzepte nutzen.
Zus\"atzlich werden das existierende JavaScript Frontend \lareJS{} und das django Backend \djangoLare{} unter Ber\"ucksichtigung dieser Richtlinien \"uberarbeitet.
Abschlie\ss{}end wird \lare{} in Form von Benchmarks mit Hilfe von \curl{} f\"ur reine \httpRequest{}s und \selenium{} zum Empfangen kompletter Webseiten, inklusive aller Ressourcen, evaluiert.
F\"ur diesen Zweck wurde eine Beispiel Web-Anwendung basierend auf PHP, Twig und den passenden \lare{} Plugins implementiert.
\\
Die Ergebnisse dieser Evaluation zeigen, dass \lare{} die Erwartungen bez\"uglich Browser Funktionalit\"at erf\"ullt und die Ladezeiten von \SinglePageApplication{}s verringert.
\end{abstract}
\selectlanguage{english}