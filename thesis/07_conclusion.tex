\section{Conclusion and future work}

We tested \lare{} to check if the browser functionality is working like on common \webApplication{}s.
The results show that those expectations regarding back and forward buttons in the browser are satisfied.
Additionally the URL changes like on normal \webApplication{}s which makes it possible to bookmark \webPage{}s.

The benchmark results show that the performance increases with \lare{}.
The effect on static pages is always quite similar and fine.

When looking at the results for dynamic pages it is a bit more complex.
The performance improvements by \lare{} vary a lot.
Load times from 3\% to 99\% relatively to normal requests display this.

Besides other reasons the biggest difference between those low and high load times is the changed content.
Requesting a very dynamic page with a not related namespace has nearly no improvements relatively to normal requests.
Being on a very dynamic page, requesting a page with only a few changes makes it possible to experience a bigger benefit of \lare{}.

What does that mean for the usage of \lare{}, when is it a good idea to use it?

Static pages have a better performance with \lare{} than without.
This makes it a good to use scenario.

When having dynamic content on a page and only a few elements change when requesting a new \webPage{} \lare{} strikes most.

\webSite{}s that are a collection of unrelated \webPage{}s are not taking a lot of benefit off \lare{}.

The more content stays from page to page, the more \lare{} scores.

Load tests with multiple users as the amount of concurrent users should be up to 10000 to be representative\cite{bib:bozdag_mesbah_vanDeursen08}.
In the future it would be good to do such a performance benchmark, to see how \lare{} effects such amounts of users.

Additionally it is recommended to test every new \lare{} backend in a way similar to the one presented in this thesis.
The results in different programming languages and \webApplication{} designs may differ from the ones displayed here.
