\section{Conclusion and future work\label{chap:conclusion}}

We implemented \phpLare{} and \twigLare{} to fulfil the browser functionality as in common \webApplication{}s.
The back and forward buttons in the browser are functioning.
Additionally the URL changes like on normal \webApplication{}s which makes it possible to bookmark \webPage{}s.
\\
The benchmark results show that the performance increases with \lare{}.
The effect on static pages is always quite satisfying and does not vary a lot.
\\
The results for dynamic pages are a bit more complex.
The performance improvements by \lare{} in this case vary a lot.
Load times from 5\% to 95\% relatively to normal requests display this.
\\
Besides other reasons the biggest difference between those low and high load times is the changed content.
Requesting a very dynamic page with a non-related namespace has nearly no improvements relatively to normal requests.
Being on a very dynamic page, requesting a page with a related namespace makes it possible to experience a bigger benefit of \lare{}.
This effect is particularly caused by avoided backend queries.
\\
%What does that mean for the usage of \lare{}, when is it a good idea to use it?
\\
Static pages have a better performance with \lare{} than without.
Accordingly it is appropriate for this scenario.
\\
When having dynamic content on a page and only a few elements change when requesting a new \webPage{} \lare{} strikes most.
\\
\WebSite{}s that are a collection of completely unrelated very dynamic \webPage{}s do not benefit much from \lare{}.
\\
As Tim Berners-Lee states in \cite{berners1998cool} URLs should not change.
When using \lare{} in an already running application it is possible to stick with the currently used URLs.
It does not need special URL patterns like Hashbang-URLs, or such.
\\
Load tests with multiple users are important to do on \webApplication{}s.
The amount of concurrent users should be up to 10000 to be representative \cite{bozdag2008performance}.
In the future it is necessary to do such a performance benchmark, to see how \lare{} behaves with such amounts of users.
\\
Additionally it is recommended to implement and test every new \lare{} backend in a way similar to the one presented in this thesis.