\section{Implementation}

Considering object-oriented programming the \emph{Lare} object builds the fundament of a \lare{} backend.
It is a singleton in the request scope.
When receiving a request the server creates it and analyzes the current \http{} header whether a namespace is set or not.
\\
The \lare{} object contains all necessary data and functions.


\subsection{\phpLare{}}
\phpLare{} is a general \lare{} backend to be used in PHP.
It builds the base of every other \lare{} module in PHP, especially for template engines.
\\
As defined before the \lare{} object is implemented as a request-scoped singleton.
This object gets created by the get\_instance() method.
To prevent multiple objects of this class the \_\_clone() and  \_\_construct() methods are replaced by empty functions.
\\
Automatically one object gets instantiated after the definition of this class.


\subsubsection{API}

\begin{itemize}
\item Lare::is\_enabled()

Returns true if the current request is a Lare request, otherwise false.
\item Lare::set\_current\_namespace(\$namespace)

Sets the namespace of the current request to \$namespace.
\item Lare::get\_current\_namespace()

Returns the namespace of the current request.
\item Lare::get\_matching\_count(\$extension\_namespace = null)

\$extension\_namespace is an optional parameter, to check the matching to a given namespace. If \$extension\_namespace is not provided, the matching will be done against the current namespace.

Returns how many parts of the namespaces (from outer to inner) are matching, before the first mismatch occurs.
\item Lare::matches(\$extension\_namespace = null)

\$extension\_namespace is an optional parameter, to check the matching to a given namespace. If \$extension\_namespace is not given, the matching will be done against the namespace of the current request.

Returns true if the whole namespace is matching, otherwise false.
\item Lare::get\_instance()

Returns the \lare{} singleton and creates it if not instantiated before.
\end{itemize}

\subsection{\twigLare{}}
Twig-Lare brings Lare functionality to the template engine Twig\footnote{http://twig.sensiolabs.org/}.
Twig is used by Symfony, a framework which is used in e.g. Drupal 8, eZPublish, phpBB and Sylius.
\\
Twig-Lare is implemented as a Twig extension.
It consists of a TwigTokenParser and the global variable lare\_current\_namespace.
The TwigTokenParser
\\
Twig\_Lare\_TokenParser\_LareExtends is the heart of the extension.
It provides the possibility to use the tag \{\% lare\_extends \%\} in the way the default twig tag \{\% extends \%\} tag is used.
Additionally it prevents multiple extend tags.
Furthermore it is ensured that it is not called inside a block tag, like seen in the default twig tag.

\begin{minipage}[c]{0.95\linewidth}
\begin{lstlisting}[caption=Example Lare Template, label=example_lare_template]


  <div id="page">
    ...
  </div>


  {{ current_lare_namespace }}

\end{lstlisting}
\end{minipage}
\\
Listing \ref{example_lare_template} shows the usage of this tag.
Using Symfony \enquote{::} is a shortcut for the default template directory.
When the current namespace is not inside \emph{Lare.Namespace} it extends to the base.twig template, because the second namespace does not match then.
When the current namespace is \emph{Lare.Namespace} lare.twig will get extended.
\\
As the namespace matching occurs on the second level in this example, the overridden block should be the according one.
For the naming in this thesis the \emph{page} block should be provided.
Inside this block a <div> container with the according ID has to be placed.

\subsubsection{API}

\begin{itemize}
\item \{\% lare\_extends \$default\_template \%\}

Extends \$default\_template like the original \{\% extends \$default\_template \%\}.
\item \{\% lare\_extends \$default\_template \$lare\_namespace \%\}

Extends ::\_\_lare.html if \$lare\_namespace is matching, otherwise it extends \$default\_template.
\item \{\% lare\_extends \$default\_template \$lare\_template \$lare\_namespace \%\}

Extends \$lare\_template if \$lare\_namespace is matching, otherwise it extends \$default\_template.
\end{itemize}


\subsection{\djangoLare{}}

\djangoLare{} was the first backend of \lare{}.
It was introduced as a single object containing logic and template tools.
After implementing \phpLare{} we decided to change the structure of the django backend towards the new segmentation.
\\
Similar to the combination of \phpLare{} and \twigLare{}, \djangoLare{} consists of a Lare object as in the PHP backend and implements the same templating tools as the Twig extension.
Still as one package it is available via \emph{pip install django-lare} command.


\subsubsection{API}

\begin{itemize}
\item \{\% lare\_extends \$default\_template \%\}

Extends \$default\_template like the original \{\% extends \$default\_template \%\}.
\item \{\% lare\_extends \$default\_template \$lare\_namespace \%\}

Extends ::\_\_lare.html if \$lare\_namespace is matching, otherwise it extends \$default\_template.
\item \{\% lare\_extends \$default\_template \$lare\_template \$lare\_namespace \%\}

Extends \$lare\_template if \$lare\_namespace is matching, otherwise it extends \$default\_template.
\end{itemize}

\todo{API of \lare{} object}

\subsection{\lareJS{}}
%\lareJS{}, as successor of jquery-pjaxr, is the frontend engine for \lare{}, using AJAX to communicate with the server.
%jQuery-pjaxr was introduced as an extended version of jquery-pjax, allowing to replace multiple containers with one request, instead of only one.
%Matching by the ID of an container it was not the job of the frontend to define which container should be replaced, but the backend with giving the correct IDs.
%\\
%Introducing \lareJS{} achieved the ability differ the current page via namespaces.
\todo{describe \lareJS{}}


\subsection{concluding remarks}
\todo{concluding remarks of \lare{}}