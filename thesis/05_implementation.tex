\section{Implementation}

\todo{decisions Lare Object}
We decided to implement the \lare{} backend in two parts.
Considering the MVC pattern we have all logic in one object, in this case the Lare object.
It is a singleton in the request scope.
When receiving a request the server creates it and analyzes if the current request is a \lare{} request and checks if there is a namespace.


\subsection{\phpLare{}}
\phpLare{} is a general \lare{} backend to used in PHP.
It builds the base of every other Lare module in PHP, especially for template engines.

\lare{} is a implemented as a request-scoped Singleton.
In PHP a singleton is implemented as a class, which provides only static methods and static attributes.

\subsubsection{API}

When including the Lare.php automatically a Singleton named Lare will be created.
\begin{itemize}
\item Lare::is\_enabled()

Returns true if the current request is a Lare request, otherwise false.
\item Lare::set\_current\_namespace(\$namespace)

Sets the namespace of the current request to \$namespace.
\item Lare::get\_current\_namespace()

Returns the namespace of the current request.
\item Lare::get\_matching(\$extension\_namespace = null)

\$extension\_namespace is an optional paramenter, to check the matching to a given namespace. If \$extension\_namespace is not given, the matching will be done against the namespace of the current request.

Returns the most specific matching namespace level.
\item Lare::matches(\$extension\_namespace = null)

\$extension\_namespace is an optional paramenter, to check the matching to a given namespace. If \$extension\_namespace is not given, the matching will be done against the namespace of the current request.

Returns true if the whole namespace is matching, otherwise false.
\end{itemize}
    
\subsection{\twigLare{}}
Twig-Lare brings Lare functionality to the template engine Twig\footnote{http://twig.sensiolabs.org/}.
Twig is used by Symfony, a framework which is used in e.g. Drupal 8, eZPublish, phpBB and Sylius.

Twig-Lare is a implemented as a Twig extension.
It consists of a TwigTokenParser, a anonymous TwigSimpleFunction and a global variable.
The TwigTokenParser Twig\_Lare\_TokenParser\_LareExtends is the heart of the extension.
It provides the possibility to use the tag \{\% lare\_extends \%\} in the way it may be used in django.
To prevent multiple extend tags, including the default twig tag \emph{\{\% extends \%\}} it throws a Error if either the default tag or \{\% lare\_extends \%\} was already used.
Additionally we ensure that it is not called inside a block tag.

\begin{minipage}[c]{0.95\linewidth}
\begin{lstlisting}


  <div id="page">
    ...
  </div>

{{ current_lare_namespace }}
\end{lstlisting}
\end{minipage}

The example above shows the usage of this tag.
When the current namespace is not inside Lare.Namespace it extends to the \:\:\_\_base.twig template because the second namespace does not match then.
But when the current namespace is \emph{Lare.Namespace}, then it extends \:\:\_\_lare.twig.

As the namespace matching occurs on the second level in this example, the overriden block should be the second, with the naming in this thesis \emph{page}.
Inside this block a <div> container with the according ID has to be placed.

\subsubsection{API}

\begin{itemize}
\item \{\% lare\_extends \$default\_template \%\}

Extends \$default\_template like the original \{\% extends \$default\_template \%\}.
\item \{\% lare\_extends \$default\_template \$lare\_namespace \%\}

Extends ::\_\_lare.html if \$lare\_namespace is matching, otherwise it extends \$default\_template.
\item \{\% lare\_extends \$default\_template \$lare\_template \$lare\_namespace \%\}

Extends \$lare\_template if \$lare\_namespace is matching, otherwise it extends \$default\_template.
\end{itemize}


\subsection{\djangoLare{}}

\djangoLare{} was the first backend of \lare{}.
It was introduced as a single object containing logic and template tools.
After implementing \phpLare{} we decided to change the structure of the django backend towards the new segmentation.

Similar to the combination of \phpLare{} and \twigLare{}, \djangoLare{} consists of a Lare object as in the PHP backend and implements the same templating tools as the Twig extension.
Still as one package it is available via \emph{pip install django-lare} command.

\subsection{\lareJS{}}
\lareJS{}, as successor of jquery-pjaxr, is the frontend engine for \lare{}, using AJAX to communicate to the server.
jQuery-pjaxr was introduced as an extended version of jquery-pjax, allowing to replace multiple containers with one request, instead of only one.
Matching by the ID of an container it was not the job of the frontend to define which container should be replaced, but the backend with giving the correct IDs.

Introducing \lareJS{} achieved the ability differ the current page via namespaces.
\todo{describe \lareJS{}}


\subsection{concluding remarks}
