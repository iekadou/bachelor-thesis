\section{Fundamentals}

To understand how \lare{} works a few technologies are required to know about.
\html{}, the markup language which is used to create \webPage{}s, is the first technology which is mentioned in this chapter.
Afterwards \httpRequest{}s are presented as the fundamental transfer protocol of the World Wide Web.
Dynamic content and synchronicity are the topics explained next, the problem addressed by \ajax{} and \singlePageApplication{}s, which are presented subsequently.
The History API, released in October 2014, is required by \lare{} and presented at the end of this chapter.

\subsection{\html{}\label{html}}
Hypertext Markup Language (\html{}) is the language which is used to create webpages.
It's specification is defined by the \gls{w3c}\footnote{http://www.w3.org/TR/html5/ (Accessed: June 24, 2015)}.
It allows to structure a web document semantically, but not to style it.
Similar to XML it consists of hierarchically structured tags.
Each tag may have attributes.
Allowed attributes are defined per tag, e.g. an anchor tag (<a>) may have an href attribute, which is not allowed on a div tag (<div>).
This href attribute defines where the anchor should lead to, when clicked on.
\\
Another type of attribute is the data attribute whose name starts with the string \enquote{data-} followed by at least one other character.
Those attributes are intended to store custom information, for which there are no more appropriate attributes or elements.
\\
There is a special attribute called \enquote{id} which defines the unique identifier of a tag.
Each value of it may not occur more than once on a \webPage{}.

\subsection{\httpRequest{}s\label{httpRequest}}
The World Wide Web has one main protocol to let \webBrowser{}s and \webServer{}s communicate, the Hypertext Transfer Protocol (\http{}), which is built on top of the Transmission Control Protocol (TCP).
TCP and so \httpRequest{}s always start with a handshake to establish a connection before data is transferred.
After this handshake, the client sends the request data to the \webServer{}.
This recognizes and interprets the request and if the requested resource is available, sends the according data back, otherwise it sends an error.
\WebApplication{}s typically render data out of a database into a \html{} template and sends it back as the response.
The browser receives this \webPage{} interprets and renders it.
Subsquently it sends \httpRequest{}s to receive the images, Cascading Style Sheets (CSS) and scripts linked in this page.
\begin{figure}[H]
\centering
\includegraphics[height=13cm]{images/http.png}
\caption[http_components]{Component and communication diagram of \http{}}
\label{fig:http_components}
\end{figure}

\noindent{}Figure \ref{fig:http_components} displays this communication of a \webBrowser{} and a \webServer{}.
On the top left you see an abstract representation of a \webPage{}.
On the right side in the middle you can see the response, in this case a complete new \webPage{}, which then replaces the previous from the top left completely.


\subsection{Dynamic content and synchronicity\label{synchronicity}}
Dynamic content in \webSite{}s is content which is not statically saved on the \webServer{}, as static content is.
Typically it gets fetched via backend services.
A backend service can be a database or other sort of API to derive information from.
Often it changes too often to save it in the filesystem of the \webServer{}.
Other reasons for this type of content can be using a content management system, or depending on data which you can not influence and has to be changed by others.
\\
The normal \httpRequest{}, which is described in \ref{httpRequest}, does not support asynchronous changes of content.
This means it is not possible to change content within an already fully loaded \webPage{}.
To retrieve new information the client has to request another full \webPage{}, including all resources and data which is necessary.
This makes this kind of \webSite{} \enquote{synchronous}.
An \enquote{asynchronous} \webSite{} is able to change content dynamically without having to load a whole new page.
\ajax{}, as shown in \ref{ajax}, is the most used technique for asynchronous \webSite{}s.

\subsection{\ajax{}\label{ajax}}
Asynchronous JavaScript and XML is a technology to implement dynamic \webPage{}s.
The JavaScript stucture XMLHttpRequest (XHR) is used to make requests to a \webServer{} without loading a full new page.
The response then is interpreted by an \ajax{}-engine.
\begin{figure}[H]
\centering
\includegraphics[height=13cm]{images/ajax.png}
\caption[ajax_components]{Component and communication diagram of \ajax{}}
\label{fig:ajax_components}
\end{figure}

\noindent{}As shown in fig. \ref{fig:http_components} a normal \http{} request by a browser forces requests always to be synchronous, due to the fact that always the whole page is replaced on every request.
\enquote{Asynchronous JavaScript and XML}, short \ajax{}, can improve that.
The data flow in AJAX is very similar to the normal browser behaviour, but is using a new component: the \enquote{\ajax{}-engine}.
The first request to a \webServer{} using \ajax{} is the complete same as one without \ajax{} with the exception, that one of the requested sources is a JavaScript, which instantiates a \ajax{}-engine.
The following requests are handled by this \ajax{}-engine, allowing to asynchronously request content from the \webServer{}.
\ajax{} can request only small parts of a website, most of the time in XML or JSON format, interprets it and then only adds, replaces or appends old content with the newly received.
Figure \ref{fig:ajax_components} shows that only one part of a \webPage{} is responded and replaced, instead of the whole page in fig. \ref{fig:http_components}.

\subsection{\SinglePageApplication{}s\label{singlePageApplication}}
A \singlePageApplication{} (SPA) is a \webApplication{} or \webSite{} that only needs one full \webPage{} load.
Beside this there is no \webPage{} loaded completely at any point in the process anymore.
Often content changes are made asynchronous and dynamically by \ajax{} in response to user actions.
A disadvantage of a lot of \singlePageApplication{}s is the lack of browser history support.
When changing the content the browser does not interpret it as a new page, but only changed content.
This leads into a missing functionality of forward- and back-buttons in browsers.
\\
Additionally a problem of SPAs is, that it has a huge impact on search engine optimization (SEO).
The \webPage{} is often rendered inside the browser, and not structured into different URLs.
Those facts make it hard for a search engines and other crawlers to discover it.

\subsection{History API}
The History API is part of the HTML5 specification by \gls{w3c}.
It describes the API for an History object which is part of the session history.
This session history enables functionality like the back and forward buttons on browsers.
Defined as a list of session history entries, it represents the browsing history.
A session history entry may be a URL or a state object and may have additional information.
\\
It is possible to influence this list via the window.history.pushState(data, title[, url]) method, which allows to add a new state into this list.
This also effects the browser's navigation bar, which gains the possibility for users to share or bookmark a \webPage{}.
\\