\section{Introduction}
At the beginning of the World Wide Web \webPage{}s were self-contained. 
And without a lot of effort they still are. 
\\
The content of a \webPage{} which is initially loaded is not changed until a new resource is requested by the user.
One big change brought the invention of Asynchronous JavaScript and XML (\ajax{}).
It introduced the possibility to change content without the need of requesting a full new \webPage{} at a different Uniform Resource Locator (URL).
The approach only loading one full \webPage{} initially and then changing its content interactively is called \singlePageApplication{}.
As presented in \cite{jonsson2009database} \singlePageApplication{}s are more user-friendly than the common designs, e.g. due to lower load times and the elimination of interruption time experienced on common \webApplication{}s.
\ajax{} is nowadays, other than mentioned in \cite{jonsson2009database}, available in nearly every browser.
As stated in \cite{herlihy2013howmany} in 2013 only 1.1\% of the internet users visiting the \webSite{} of the UK government did not get their JavaScript enhancements.
\\
As mentioned in \cite{jonsson2009database} a disadvantage of \ajax{} is that users are not able to save their websites as a bookmark, because while surfing on this page, the URL never changes.
Additionally this lack of URLs for subpages leads to a contradiction to the current HTTP standard as defined in \cite{fielding1999hypertext}.
A resource is defined there as \enquote{A network data object or service that can be identified by a URI.}
A page which can only be accessed by \ajax{} and not via a normal URL does not fulfil this requirement.
\\
Another implication stated in \cite{jonsson2009database} is that the functionality of the back button in browsers is often not given.
Also \cite{estrada2011take} complains that \enquote{...previously seen pages are not always reachable through it.}.
Additionally \ajax{} driven \webApplication{}s may require multiple small server calls which might produce performance implications.
\\
\SinglePageApplication{}s have another problem in the current time. 
As mentioned in \cite{matter2008ajax} \ajax{} websites are difficult to examine for search engines and other crawlers, due to several challenges.
\\
As Google is the most used search engine in the World Wide Web they have a design pattern\footnote{https://developers.google.com/webmasters/ajax-crawling/docs/getting-started (Accessed: Juli 22, 2015)} for implementing a crawlable \ajax{} web application.
In this guideline it is recommended to have snapshots of every \webPage{} available under non-user-friendly \gls{url}s, called Hash-Bang URLs.
This means additional maintenance effort for \webSite{}s, especially when they are very dynamic.
\\
Even though Google can interpret JavaScript generated content since May 2014\footnote{http://googlewebmastercentral.blogspot.no/2014/05/understanding-web-pages-better.html} they still give the advice to degrade graceful when it comes to JavaScript compatibility.
\\
In this thesis we will improve and evaluate the performance of a new technology called \lare{}.
Together with Stephan Gro{\ss} the author developed PJAXR the predecessor of \lare{} for using \ajax{} with it's advantages but trying to avoid the disadvantages explained before.
\\
The result of this cooperation was a frontend-side implementation called \emph{jquery-pjaxr}.
As PJAXR also needed a backend to be implemented, the author developed the first PJAXR backend called \emph{django-pjaxr}.
\\
The libraries \lareJS{} and \djangoLare{} are introduced in this thesis as the successors of \emph{jquery-pjaxr} and \emph{django-pjaxr}.
Additionally in this thesis the author will introduce a new \lare{} backend for PHP, \emph{\phpLare{}} and \emph{\twigLare{}} an extension for the \twig{}\footnote{http://twig.sensiolabs.org/ (Accessed: June 16, 2015)} template-engine.
\\
Later \lare{} will be evaluated to see whether the expectations regarding browser functionality are satisfied.
Furthermore the single user performance will be tested through benchmarks using \curl{} and \selenium{} in combination with the \webdriver{} API, to see whether \lare{} is faster than normal web requests.
