\section{Introduction}
At the beginning of the World Wide Web \webPage{}s were self-contained. 
And without a lot of effort they still are. 

The content which is initially loaded is not changed until a new URL is requested by the user.
One big change brought the invention of \ajax{}. It introduces the possibility to change content without the need of requesting a new URL.
This approach only loading one website initially and then changing its content interactively is called \singlePageApplication{}.
\SinglePageApplication{}s are more user-friendly than the common designs, e.g. due to lower load times in combination with not being able to indicate clearly that a new website is being loaded.
A big disadvantage is that users are not able to save their websites as a bookmark, because while surfing on this page, the URL never changes.
\ajax{} is available in nearly every browser which is a reason that a lot of different frameworks gain this functionality, improving and enhancing it.

\SinglePageApplication{}s have a problem in the current time: Search engines and other crawlers trying to examine websites will find nothing more than content which was provided initially. Every further change of content is not easily accessible.
As Google is the most used search engine in the World Wide Web they have a design pattern\footnote{https://developers.google.com/webmasters/ajax-crawling/docs/getting-started} for implementing a crawlable \ajax{} web application. 
In this guideline it is recommended to have snapshots available under non-user-friendly \gls{url}s, called Hash-Bang URLs.

In this thesis we will improve and evaluate the performance of a new technology called \lare{}. 
Together with Stephan Gro{\ss} I developed PJAXR, a technique for using \ajax{} with it's advantages but trying to avoid the disadvantages explained before.
The result of this cooperation was a frontend-side implementation called \emph{jquery-pjaxr}.
As PJAXR also needs a backend to be implemented, I developed the first PJAXR backend called \emph{django-pjaxr}.

The libraries \lareJS{} and \djangoLare{} are introduced in this thesis as the successors of \emph{jquery-pjaxr} and \emph{django-pjaxr}.
Additionally in this thesis I will introduce a new \lare{} backend for PHP, \emph{\phpLare{}} and \emph{\twigLare{}} an extension for the \twig{}\footnote{http://twig.sensiolabs.org/} template-engine.
