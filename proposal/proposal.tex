\documentclass[f,bachelor,binding,twoside,palatino]{WeSTthesis}
% Please read the README.md file for additional information on the parameters and overall usage of WeSTthesis

\usepackage[ngerman, english]{babel}         % English and new German spelling
\usepackage[utf8]{inputenc}                 % correct input encoding
\usepackage[T1]{fontenc}                    % correct output encoding
\usepackage{graphicx}                       % enhanced support for graphics
\usepackage{tabularx}                       % more flexible tabular
\usepackage{amsfonts}                       % math fonts
\usepackage{amssymb}                        % math symbols
\usepackage{amsmath}                        % overall enhancements to math environment
\usepackage[babel,english=british]{csquotes}
\usepackage{enumitem}
\usepackage{url}
\usepackage{glossaries}
\usepackage{xcolor}
\makeglossaries

\def \ajax {AJAX}
\def \pjaxr {PJAXR}
\def \jqueryPjaxr {jquery-pjaxr}
\def \httpRequest {HTTP request}
\def \singlePageApplication {single-page application}
\def \SinglePageApplication {Single-page application}

\usepackage{glossaries}

\newglossaryentry{ajax}
{
  name=AJAX,
  description={Asynchronous JavaScript and XML \footnote{\url{https://developer.mozilla.org/de/docs/AJAX}}}
}

\newglossaryentry{djangoLare}
{
  name=django-lare,
  description={The \lare{} backend for django}
}

\newglossaryentry{dynamicWebPage}
{
  name=dynamic web page,  
  description={A dynamic web page is a web page that is generated by a web application on a web server before it gets delivered.}
}

\newglossaryentry{hijax}
{
  name=HIJAX,
  description={Hijax}
}

\newglossaryentry{html}
{
  name=HTML,
  description={HTML stands for Hyper Text Markup Language and is the language that is used in the Web.}
}

\newglossaryentry{httpRequest}
{
  name=HTTP request,
  description={Hypertext Transfer Protocol requests build the foundation for data communication in the World Wide Web.}
}

\newglossaryentry{lareJS}
{
  name=lare.js,
  description={lare.js is the \lare{} frontend and the \ajax{} engine for \lare{}.}
}

\newglossaryentry{phpLare}
{
  name=PHP-lare,
  description={PHP-lare is the \lare{} backend for PHP.}
}

\newglossaryentry{lare}
{
  name=Lare,
  description={Lightweight asynchronous replacement engine is a technology for stateful \singlePageApplication{}s.
  It consists of a \lare{} frontend as ajax engine, and a \lare{} backend which is plugged into the web application.}
}

\newglossaryentry{singlePageApplication}
{
  name=single-page application,
  description={A single-page application is a web application or web site that retrieves one full \webPage{}.
  Beside this first page load, the web site is not loaded completely at any point in the process anymore.
  Often content changes are made asynchronous and dynamically by \ajax{} in response to user actions.}
}

\newglossaryentry{staticWebPage}
{
  name=static web page,
  description={A static web page is a web page that is delivered exactly as stored in the web server's file system.}
}

\newglossaryentry{twig}
{
  name=Twig,
  description={Twig}
}

\newglossaryentry{twigLare}
{
  name=Twig-lare,
  description={Twig-lare is the \lare{} backend for Twig and uses \phpLare{}.}
}

\newglossaryentry{url}
{
  name=URL,
  description={A Uniform Resource Locator identifies and defines the location of a resource, e.g. a \webPage{}.}
}

\newglossaryentry{w3c}
{
  name=W3C,
  description={The World Wide Web Consortium (W3C) is an international community where Member organizations, a full-time staff, and the public work together to develop Web standards. Led by Web inventor Tim Berners-Lee and CEO Jeffrey Jaffe, W3C's mission is to lead the Web to its full potential.\footnote{http://www.w3.org/Consortium/}}
}

\newglossaryentry{webApplication}
{
  name=web application,
  description={A web application is a application that generates \webPage{}s.}
}

\newglossaryentry{webPage}
{
  name=web page,
  description={A web page is a single document and part of a web site. 
  Every page should be accessible over the Internet and has it's own URL.
  A web browser can retrieve web pages by making \httpRequest{}s and it can render them afterwards.}
}

\newglossaryentry{webSite}
{
  name=web site,
  description={A web site is a collection of \webPage{}s that are linked to each other.}
}



\author{Jonas Braun}

\title{\pjaxr{} - A new technology for stateful \singlePageApplication{}s}

\degreecourse{Informatik}

\firstreviewer{Prof. Dr. Steffen Staab}
\firstreviewerinfo{Institute for Web Science and Technologies}

\secondreviewer{René Pickhardt}
\secondreviewerinfo{Institute for Web Science and Technologies}


\begin{document}

% optional: change document language from ngerman to english
% \selectlanguage{english}

\maketitle %prints the cover page  an empty page if two-sided print
\pagenumbering{roman}

\tableofcontents

\varclearpage

% list of figures
% \listoffigures
% \varclearpage

\pagenumbering{arabic}

% beginning of the actual text section
\newcommand\todo[1]{\textcolor{red}{#1}}

\section{Introduction}
    At the beginning of the World Wide Web websites had a similarity, they were self-contained. The content which was initially loaded was not changed until a new URL was requested by the user.
    One big change was the integration of Flash into browsers. It introduced the possibility to render animations and changing contents without the need of requesting a new URL.
    This approach only loading one website initially and then changing its content interactively is called \singlePageApplication{}.
    \SinglePageApplication{}s are more user-friendly than the common designs, e.g. due to lower load times in combination of not being able to indicate clearly that a new website is being loaded.
    A big disadvantage of those are that users are not able to save their websites as a bookmark, because while surfing on this page, the URL never changes.
    Due to some disadvantages of Flash and an increasing usage of JavaScript and it's development, more and more pages started to use \ajax{}, a method to asynchronously request content and update the website's content. This technique introduces the ability to implement a \singlePageApplication{} using \ajax{}, which is available in nearly every browser. A lot of different frameworks gain this functionality, improving and enhancing it.

    Both of those approaches have a problem in the current time: Search engines and other crawlers trying to examine websites will find nothing more than content which was provided initially. Every further change of content is not easily accessible.
    As Google is the most used search engine in the World Wide Web they have a design pattern\footnote{https://developers.google.com/webmasters/ajax-crawling/docs/getting-started} for implementing a crawlable \ajax{} web application. In this guideline it is recommended to have snapshots available under an non-user-friendly URL.


    \todo{change this}  
    In this thesis we will evaluate if \pjaxr{} is able to make a web application crawlable without using a special second endpoint for crawlers or implementing specific design patterns for search engines.


\section{Related Work}
  \ajax{} is a widely used technique in the internet to build web applications because of the UX improvements it brings.
  In \cite{roodt06} is mentioned that \ajax{} applications have a better usability than non-\ajax{} websites.
  The same conclusion is made in \cite{klugeKarglWeber07}, despite of the lack of browser navigation support.
  Beside the navigation problem another disadvantage is, as presented in \cite{mesbah09}, crawling \ajax{} applications is not trivial.
  One solution of this task is finding clickables and navigating to every page found by this.
  Nevertheless \cite{mesbah09} states also that this only generates a snapshot of the full application.
  This difficulty of crawling those websites is additionally presented by search engines, avoiding this problem.\cite{matter08}, page 81.


    \todo{Anforderung vs. Realisierungskonzepte vs. Implementierung}  

\section{State of the art}
    \todo{remove flash}  
  At the beginning of the World Wide Web the most common technique was to load every website completely. Changing content was not possible.
    
  With the rise of Flash in the early 2000's, it was possible to have \singlePageApplication{}s. It was possible to send asynchronous requests out of Flash to load the needed content individually.
  The engineers of Flickr found a feature of JavaScript called XMLHTTPRequest and invented Asynchronous JavaScript and XML, short \ajax{}. It was now possible to use JavaScript to load content without the need to reload full pages.
  This technique and iOS' full featured browser incl. JavaScript and no support of Flash started the fall of Flash. From 2007 on \ajax{} was the technique to use for \singlePageApplication{}s.
  
\section{Fundamentals}
  To understand how \ajax{} and similar techniques work, you have to understand how \httpRequest{}s work:
  First a browser interprets the URL and requests the path, e.g. "/index.html" from the host, e.g. "google.de" via the protocol, e.g. "http".
  Then the webserver analyses the request and responds with the content, corresponding to the requested path.
  To display the requested website, the browser interprets the response and renders the content. 
  Images, JavaScripts and Stylesheets which are linked inside the response are retrieved and interpreted the same way on further \httpRequest{}s.

  The first response is typically a HTML file.
  Starting with the used document type those files have a XML like structured hierarchy.
  Every item inside is a semantic tag, which can have attributes like a class or an ID.

  \SinglePageApplication{}s using \ajax{} aim on changing those tags dynamically without having to load a full new page.
  When e.g. clicking a link, JavaScript will prevent the default browser action for clicking a link, a request of a new website, but it will do it internally.
  It requests a special URL to retrieve some data, most often JSON, which is then interpreted and rendered, often into a frontend template, using JavaScript.

  Introducing HTML 5 on 28th October 2014 W3C released a new standard for HTML and associated APIs. 
  This release introduced officially the History API.
  
\section{\pjaxr{}}
  \pjaxr{}, based on \ajax{}, uses the History API of HTML 5.
  Part of that are the JavaScript functions pushState, replaceState and history state objects.
  To load a page, the first request is a normal \httpRequest{} followed by a JavaScript initializing the \jqueryPjaxr{} module.
  Further requests to the same host are then interpreted by \jqueryPjaxr{}.
  The content delivered by a \pjaxr{} backend is interpreted and replaces tags with matching IDs.
  \pjaxr{} in combination with the History API makes it then possible to update the content and change the URL, like it would be made by normal requests.
  Also the back- and forward-buttons in a browser will work as it were a full request.
  
  Additionally when requesting an URL, which is requested by \jqueryPjaxr{}, a full page will be responded.
  This is possible with using a \pjaxr{} backend at the webserver and a well structured HTML hierarchy.
  \ajax{} usually requests URLs pointing to JSON or other formatted data, not representing a full website, even on non-\ajax{} requests.
  This makes it possible for users to save bookmarks of the website with the current content.

\section{Implementation}
  Part of the thesis and to evaluate \pjaxr{} a sample web application will be implemented, which will provide non-\ajax{}, \ajax{}-based and \pjaxr{}-based endpoints.
  This will especially include an implementation of a \pjaxr{} backend for PHP. As another part \jqueryPjaxr{} will be adjusted to fulfil the requirements for a PHP backend.

\section{Evaluation}
  As one traditional testing model, we will evaluate the \pjaxr{} sample application via blackbox tests.
  Testing \ajax{} is not trivial due to multiple programming- and markup-languages influencing it. 
  One possibility to test web applications, as suggested in \cite{lundmark11}, is Selenium\footnote{http://www.seleniumhq.org/}.
  With this tool it is possible to generate automated tests for web applications.
  
  To evaluate whether the application is crawlable or not is the biggest criteria of \pjaxr{}.
  One way to crawl \ajax{} web applications, recommended in \cite{crawljax:tweb12} is to use Crawljax\footnote{http://crawljax.com/}. 
  It explores \ajax{}-based web applications by following every link recursively and saving the associated content. In this thesis the three endpoints of the sample project will be crawled by Crawljax to see whether all endpoints provide the same content or not.
  
  Another way to evaluate whether \pjaxr{} fulfils its goals, is testing if the Googlebot\footnote{http://google.com/bot.html} will discover all the content provided.
  Again, all three endpoints will be tested to check, if all data is found by this technique.
  While Crawljax is intended to find not easily accessible content, Googlebot is intended to find content, matching design patterns\footnote{https://developers.google.com/webmasters/ajax-crawling/} by Google. This fact makes it more challenging for \pjaxr{}, not implementing these, to have good results in this test.

\section{Contents}

\begin{enumerate}
  \item Introduction
  \begin{enumerate}[label*=\arabic*.]
    \item Background and motivation
    \item Goals of this thesis
    \item Thesis outline
  \end{enumerate}
  \item Fundamentals
  \begin{enumerate}[label*=\arabic*.]
    \item \httpRequest{}
    \item HTML
    \item Single-Page applications
    \item \ajax{}
    \item History API
  \end{enumerate}
  \item State of the art
  \begin{enumerate}[label*=\arabic*.]
    \item Hijax
    \item Hash-Bang URLs
  \end{enumerate}
  \item \pjaxr{}
  \begin{enumerate}[label*=\arabic*.]
    \item Introduction
    \item Concept
    \item Functionality
    \item Usage
    \item \pjaxr{}-backends
  \end{enumerate}
  \item Implementation
  \begin{enumerate}[label*=\arabic*.]
    \item Sample PHP web application
    \item PHP \pjaxr{}-backend
    \item \jqueryPjaxr{} adjustments
    \item Concluding remarks
  \end{enumerate}
  \item Evaluation
  \begin{enumerate}[label*=\arabic*.]
    \item Selenium
    \item Crawljax
    \item Googlebot
    \item Results
    \item Concluding remarks
  \end{enumerate}
  \item Conclusion and future work

\end{enumerate}

\printglossary

\begin{thebibliography}{9}

\bibitem{roodt06}
  Youri op't Roodt (2006).
  \emph{The effect of Ajax on performance and usability in web environments}.
  Master Thesis. University of Amsterdam

\bibitem{klugeKarglWeber07}
  Kluge, Jonas and Kargl, Frank and Weber, Michael (2007).
  \emph{The effects of the AJAX technology on web application usability}.
  WebIST 2007, Barcelona

\bibitem{mesbah09}
  Mesbah, Ali (2009).
  \emph{Analysis and Testing of Ajax-based Single-page Web Applications}.
  Ph.D. Thesis. TU Delft

\bibitem{matter08}
  Duda, Cristian and Frey, Gianni and Kossmann, Donald and Matter, Reto and Zhou, Chong (2009).
  \emph{AJAX Crawl: Making AJAX Applications Searchable}.
  Data Engineering, 2009. ICDE '09, Shanghai
  
\bibitem{lundmark11}
  Lundmark, Simon (2011)
  \emph{Automatic Testing of Modern Web Applications in an Agile Environment}
  Bachelor Thesis, Stockholm

\bibitem{crawljax:tweb12}
  Mesbah, Ali and van Deursen, Arie and Lenselink, Stefan
  \emph{Crawling {Ajax}-based Web Applications through Dynamic Analysis of User Interface State Changes}
  ACM Transactions on the Web (TWEB) 2012
  
\end{thebibliography}

\end{document}
